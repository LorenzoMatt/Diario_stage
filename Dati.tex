%----------------------------------------------------------------------------------------
%   USEFUL COMMANDS
%----------------------------------------------------------------------------------------

\newcommand{\dipartimento}{Dipartimento di Matematica ``Tullio Levi-Civita''}

%----------------------------------------------------------------------------------------
% 	USER DATA
%----------------------------------------------------------------------------------------

% Data di approvazione del piano da parte del tutor interno; nel formato GG Mese AAAA
% compilare inserendo al posto di GG 2 cifre per il giorno, e al posto di 
% AAAA 4 cifre per l'anno
\newcommand{\dataApprovazione}{Data}

% Dati dello Studente
\newcommand{\nomeStudente}{Lorenzo}
\newcommand{\cognomeStudente}{Matterazzo}
\newcommand{\matricolaStudente}{1195360}
\newcommand{\emailStudente}{lorenzo.matterazzo@studenti.unipd.it}
\newcommand{\telStudente}{+ 39 366 223 6401}

% Dati del Tutor Aziendale
\newcommand{\nomeTutorAziendale}{Fabio}
\newcommand{\cognomeTutorAziendale}{Pallaro}
\newcommand{\emailTutorAziendale}{f.pallaro@synclab.it}
\newcommand{\telTutorAziendale}{+ 39 333 136 8500}
\newcommand{\ruoloTutorAziendale}{}

% Dati dell'Azienda
\newcommand{\ragioneSocAzienda}{Sync Lab S.r.l.}
\newcommand{\indirizzoAzienda}{Galleria Spagna, 28, 35129 Padova PD}
\newcommand{\sitoAzienda}{https://www.synclab.it/}
\newcommand{\emailAzienda}{info@synclab.it} 
\newcommand{\partitaIVAAzienda}{P.IVA 07952560634} %TODO

% Dati del Tutor Interno (Docente)
\newcommand{\titoloTutorInterno}{Prof.}
\newcommand{\nomeTutorInterno}{Paolo}
\newcommand{\cognomeTutorInterno}{Baldan}

\newcommand{\prospettoSettimanale}{
     % Personalizzare indicando in lista, i vari task settimana per settimana
     % sostituire a XX il totale ore della settimana
    \begin{itemize}
        \item \textbf{Prima Settimana (40 ore)}
        \begin{itemize}
            \item Incontro con persone coinvolte nel progetto per discutere i requisiti e le richieste relativamente al sistema da sviluppare;
            \item Verifica credenziali e strumenti di lavoro assegnati;
            \item  Ripasso Java Standard Edition e tool di sviluppo (IDE ecc.);
            \item Studio teorico dell’architettura a microservizi: passaggio da monolite a microservizi con pro e contro;
            \item Ripasso principi della buona programmazione (SOLID, CleanCode);
            \item Ripasso Java Standard Edition.
        \end{itemize}
        \item \textbf{Seconda Settimana - (40 ore)} 
        \begin{itemize}
            \item Studio teorico dell’architettura a microservizi: passaggio da monolite ad architetture a microservizi;
            \item Studio teorico dell’architettura a microservizi: Api Gateway, Service Discovery e Service Registry, Circuit Breaker e Saga Pattern;
            \item Studio Spring Core/Spring Boot.
        \end{itemize}
        \item \textbf{Terza Settimana - (40 ore)} 
        \begin{itemize}
            \item Studio servizi REST e framework Spring Data REST;
            \item Studio ORM, in particolare il framework Spring Data JPA.
        \end{itemize}
        \item \textbf{Quarta Settimana - (40 ore)} 
        \begin{itemize}
            \item Studio ORM, in particolare il framework Spring Data JPA.;
        \end{itemize}
        \item \textbf{Quinta Settimana - (40 ore)} 
        \begin{itemize}
            \item Studio della piattaforma SportWill esistente;
            \item Analisi nuova funzionalità da implementare.
        \end{itemize}
        \item \textbf{Sesta Settimana - (40 ore)} 
        \begin{itemize}
            \item Implementazione del nuovo servizio.
        \end{itemize}
        \item \textbf{Settima Settimana - (40 ore)} 
        \begin{itemize}
            \item Implementazione del nuovo servizio.
        \end{itemize}
        \item \textbf{Ottava Settimana - Conclusione (40 ore)} 
        \begin{itemize}
            \item Considerazioni e collaudi finali.
        \end{itemize}
    \end{itemize}
}

% Indicare il totale complessivo (deve essere compreso tra le 300 e le 320 ore)
\newcommand{\totaleOre}{320}

\newcommand{\obiettiviObbligatori}{
	 \item \underline{\textit{O01}}: Acquisizione competenze sulle tematiche sopra descritte;
	 \item \underline{\textit{O02}}: Capacità di raggiungere gli obiettivi richiesti in autonomia seguendo il cronoprogramma;
	 \item \underline{\textit{O03}}: Portare a termine l’implementazione dei microservizi richiesti con una percentuale di superamento pari al 80.
	 
}

\newcommand{\obiettiviDesiderabili}{
	 \item \underline{\textit{D01}}: Portare a termine l’implementazione dei microservizi richiesti con una percentuale di superamento pari al 100.
}

\newcommand{\obiettiviFacoltativi}{
	 \item \underline{\textit{F01}}: Utilizzo della containerizzazione per portare tutti i microservizi su Docker.
}